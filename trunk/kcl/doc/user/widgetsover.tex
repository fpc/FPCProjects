%===============================================================================
% KCL User's Guide
% (c) 2000 Sebastian Guenther, sg@freepascal.org
% $Id$
%===============================================================================

\chapter{Widgets overview}

%-------------------------------------------------------------------------------

\section{What are widgets?}
In KCL, the term "Widget" refers to a visual user interface element; this can be
a control element such as a button or label, or even a whole window (a main
windows or dialog box, for example.)
All widgets derive from TKCLComponent, i.e. they are normal FCL components
with the additional ability to support events and event filtering.
\\ The widget classes share a lot of common properties and methods, therefore
they have a common base class called \texttt{TWidget}. All widgets support the
following:
\begin{itemize}
\item A handle for the underlying object of the library which is used by the
      current KCL target. (Example: For Windows, this would be a HWND,
      for GTK this would be a PGtkWidget)
\item A parent widget
\item A text which is displayed on the widget (property "\texttt{Text}")
\item Current size, minimum size, maximum size, size mode
\item Methods for adding and removing child widgets
\item Methods for redrawing a whole or a part of a widget
\item Handling of the input focus
\item Different input states such as deactivated (property "\texttt{Enabled}")
\item Hints, i.e. short descriptive texts that pop up when the user moves the
      mouse cursor over the widget (properties "\texttt{Hint}" and
      "\texttt{ShowHint}")
\item Some basic events: \texttt{OnCreate} is called when the widget object
      of the underlying widget library has been created; \texttt{OnFocusIn} and
      \texttt{OnFocusOut} are called when the widgets receives or looses the
      input focus.
\end{itemize}

\section{Ownership}
All widgets have an owner (property "\texttt{Owner}") and a parent widget
(property "\texttt{Parent}").
\\ The owner is already introduced in \texttt{TWidget}'s ancestor
\texttt{TComponent}, this is the logical owner of the widget seen as a
component. The main function of this ownership is the ability to stream all
components which are belonging together at once and to free all children
components when the owner component is being destroyed.
\\ The parent, which exists only for widgets, is the 'visual' parent of the
widget. A menu item belongs to its menu, a layout widget may be the parent
of a button and so on.
\subsection{Streaming}
The streaming support code in KCL is able to stream components together with
its subcomponents. It tries to store the components in a hierarchical manner;
for widgets the hierarchy is given through the 'visual' hierarchy, i.e. the
parent/child relationships. (Internally all relations are defined using the
\texttt{TComponent} methods \texttt{HasParent} and \texttt{GetParentComponent},
\texttt{TWidget} overwrites both methods so that they return the \texttt{Parent}
property of the widget.)

\section{Size and resizing}
One of the most complex basic mechanisms in KCL is the size and resizing
algorithm.
\\ Basically, all widgets have the following properties which influence the
size:
\begin{itemize}
\item \texttt{SizeMode}: This property controls how the initial size of a widget
      after its creation is calculated, and how the widget behaves when the
      widget itself or its parent is being resized. The following values are
      defined for the size mode:
      \begin{itemize}
      \item \texttt{sizeAuto} (default)
      \item \texttt{sizeManual}
      \item \texttt{sizeFixed}
      \item \texttt{sizeFill} (only valid for widgets which have a parent)
      \end{itemize}
\item \texttt{Constraints} (Type \texttt{TSizeConstraints}): Its properties
      \texttt{MinWidth} and \texttt{MinHeight} represent the minimum size of the
      widget; both values default to "0", which is interpreted as the default
      size of the widget.
      A widget can not get smaller than its default size, regardless of the
      values of these properties!
      \\ Properties \texttt{MaxWidth} and \texttt{MaxHeight}: This is the
      maximum size of the widget. -1 x -1 means the default size of the widget;
      0 x 0 (which is the default) is interpreted as infinite, i.e. the maximum
      size of the widget is not restricted.
\item \texttt{Width} and \texttt{Height}: Finally, this is the current size of
      a widget. Changing the values of these properties may have different
      results, depending on the current size mode: (all sizes are clipped
      against the Constraints property of the widgets, of course)
      \\ For the manual size mode and for the fixed size mode, the widget size
         is set directly.
      \\ For the auto size mode, changing these values will set the size mode
         to \texttt{sizeManual}, the current size will be adjusted.
      \\ For the fill size mode, changes of these values will be ignored.
\end{itemize}
All properties can be changed without any effect, as long as the widget creation
has not been finished. \texttt{OnFinishCreation} will have to calculate the
default size of the widget and to store it in the protected variables
\texttt{TWidget.DefWidth} and \texttt{TWidget.DefHeight}.
Finally, as soon as \texttt{TWidget.\-OnFinishCreation} is executed or any of the
properties listed above is changing, the method \texttt{TWidget.SizesChanged}
will be called. This method adjusts the real size of the widget, if necessary.
\\ Furthermore, when a widget has changed its size it calls the method
\texttt{RecalcLayout} for its parent and then for all of its child widgets.
Notifying the parent is important for auto-sizing parents, and it even may
influence the minimum and/or maximum size of the parent. Notifying the children
enables space-filling children to adjust their sizes.
\\ To support the size calculations, there is a protected virtual method called
\texttt{TWidget.\-GetReal\-MinMaxSize}; this method calculates the "real" minimum
and maximum size in pixels, i.e. the special values "0" and "-1" for the sizes
are set to the right currently valid values; additionally the widgets clip these
values against its children sizes. (example: a layout widget with maximum size
200x60 which holds a button with a maximum size of 50 x 50 and which has a
border of 10 pixels on each side, will return 70 x 60 as real maximum size.)
KCL widgets should not change their size properties only because the current
situation violates some of the sizing rules, as in the above example.