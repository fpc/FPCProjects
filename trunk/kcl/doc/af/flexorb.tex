\documentclass{report}
\usepackage{html}
\usepackage{thumbpdf}
\usepackage{fpc}
\usepackage{times}
\begin{document}
\author{Sebastian Guenther}
\title{FlexORB User's Guide and Reference}
\docversion{Version 0.01}
\maketitle
\tableofcontents

%===============================================================================
\chapter{Introduction}

%-------------------------------------------------------------------------------

\section{General}
FlexORB is a lightweight and flexible ORB for CORBA. Its main target is the \fpc
Compiler (FPC). It is written in \fpc, and it is built around the special
features of this compiler (e.g. its support for AnsiStrings).
\newline Just for clarification:
\begin{itemize}
\item \textbf{CORBA}: Common Object Request Broker Architecture; defines a
      platform and language independent object model and some important services
      with client/server environments as the main target.
\item \textbf{CORBA object}: Object in the sense of OOP, which has a
      well-defined interface to its environment.
\item \textbf{IDL}: Interface Description Lanuage, declarative language derived
      from C++ which is used to describe the interfaces of CORBA objects in a
      platform and language independent manner. You use an \textit{IDL compiler}
      to create mappings to a specific programming language (here mappings means
      e.g. header files in C/C++ or units in Pascal).
\item \textbf{ORB}: Object Request Broker, a CORBA object which provides the basic
      services, e.g. all methods needed for bootstrapping the whole CORBA
      system.
\item \textbf{GIOP}: General Inter-ORB protocol. This is a well-defined binary
      protocol which can be used by different ORBs to communicate with each
      other.
      Normally ORBs use \textit{bridges} to support objects which run on a
      different ORB. This process is transparent to the developer.
\item \textbf{IIOP}: A specific GIOP application: IIOP is an extension to GIOP
      which defines how ORBs communicate using an TCP/IP connection. Using IIOP,
      it is possible that different ORBs run on different machines within a
      network.
\end{itemize}
The most important features of FlexORB are:
\begin{itemize}
\item Small memory footprint
\item Extensibility: Provides internal interfaces to extend the ORB with
      features which are not available as default.
      (hence the name \textit{Flex}Orb).
\item IIOP bridge, which enables FlexORB to communicate with other ORB's
      (possibly running on different machines) and objects written in other
      programming languages or scripting languages.
\item Direct support of C and C++ with some special libraries to make it
      possible to implement CORBA objects in C/C++. GIOP/IIOP in conjunction
      with a C/C++ ORB can do this job, too, but such solutions are
      \textit{much} slower.
\item Provides internal interfaces to extend the ORB with features which are
      not available as default. (hence the name \textit{Flex}Orb).
\end{itemize}

%-------------------------------------------------------------------------------

\section{Design Decisions}
The most important thing about FlexORB is that it \textit{cannot} be fully CORBA
compliant. The CORBA specification requires that a compliant implementation
supports at least one of the defined language mappings. But unfortunately there
is no mapping for Object Pascal, so FlexORB uses its own mapping. This can be
seen as an important advantage: FlexORB does not really try to be CORBA
compliant as \textit{close} as possible, but it tries to be as \textit{simple}
as possible. It is not the goal of FlexORB to provide all defined CORBA
features and services; it is only meant as a way to make FPC more open to the
rest of the OOP world via the GIOP/IIOP bridge.
\subsection{Client view}
The following decisions influence the way how clients of CORBA objects see the
ORB and the objects:
\begin{itemize}
\item As a general rule the CORBA objects shall look to the clients almost
      the same as native objects. The exact mapping which provides this
      requirement is described in detail in chapter \ref{ChapterMapping} on
      page \pageref{ChapterMapping}.
\item The ORB can be seen as an object with a simple public interface. It has to
      be initialized via the standard \texttt{ORB\_Init} function, which
      returns a reference to the ORB as a pointer to an \texttt{IOrb} interface.
\item All object references are simply pointers to an interface
      object. It does not distinguish between a reference to a CORBA object and
      directly created instances.
      \footnote{This is different to the C++ mapping: In C++ you can create
      objects directly on the stack, whereas \fpc objects and interfaces
      can only exist as a memory object with pointers attached to them.
      Therefore, the C++ mapping uses two different types per object interface,
      but the FlexORB mapping uses only one - the reference which is native to
      Object Pascal.}
      Interfaces cannot have constructors or destructors, the only way to
      create new object instances is to call 
\item Object references have the same maximum life span as the ORB; i.e. you
      cannot store an object reference (or its string equivalent using
      \texttt{ORB.object\_to\_string}) within a file and retrieve it in
      another session of the same program. All object references are only valid
      within the FlexORB instance which created the object.
      \newline This is the same semantic as with standard Object Pascal objects.
\end{itemize}

\subsection{Implementator's view}
For implementators of CORBA objects (in C/S environments the CORBA
objects can be seen as \textit{servers}). it is important that they can use the ORB
as simple as possible. The following design decisions try to support this:
\begin{itemize}
\item The ORB tries to be as compliant as possible, but it only supports the
      methods which are needed by the other parts of FlexORB.
\item The defined object adapters (BOA (Basic Object Adapter) and POA (Portable
      Object Adapter)) provide services which are not applicable to FlexORB.
      Normally it is recommended to implement CORBA objects using the POA as
      an implementation helper. The reason is that the POA provides a common
      API for object implementations, regardless of the ORB used; it makes the
      implementations portable between different ORBs. As there is no other
      ORB for \fpc but FlexORB yet, and as there is no CORBA compliant
      Pascal mapping in CORBA, this argument does not apply here. Supporting
      the full POA interface would increase the complexity of FlexORB
      tremendously.
      \newline FlexORB object implementations use a much simpler base class
      than the POA instead, which is called \texttt{TCorbaObject}.
\end{itemize}

%===============================================================================
\chapter{CORBA Mapping to \fpc}
\label{ChapterMapping}

This chapter discusses how the basic CORBA structures and services are being
mapped to the \fpc language. It also defines an IDL to \fpc mapping, and the
reverse mapping from \fpc source to IDL.

%-------------------------------------------------------------------------------

\subsection{Overview}
The basic mapping of CORBA and IDL types follows some quite simple rules. 
\begin{itemize}
\item Strings are always represented as \texttt{AnsiStrings}. "\texttt{in}"
      arguments of methods are translated to \texttt{const AnsiString}.
\item Object interfaces directly map to the \fpc \texttt{interface} construct.
    \marginpar{See chapter \ref{ChapterFPCBeta} for FPC beta remarks}
\end{itemize}

%-------------------------------------------------------------------------------

\subsection{IDL mapping}
Mapping all possible members of an IDL definition is somewhat difficult, as it
is possible in IDL to declare certain type definitions within an interface
declaration - this is not possible in \fpc at all. The solution is to insert
such declarations in front of the interface declaration, additionally the name
of the interface is added in front of the name of the declaration.
\newline A simple example:
\begin{verbatim}
module Test
{
  interface MyInterface {
    typedef struct TestStruct {
      int x, y;
    };

    void setStruct(in TestStruct tstr);
  };
};
\end{verbatim}
would map to:
\begin{verbatim}
unit Test;
interface
type
  TMyInterface_TestStruct = record
    x, y: Integer;
  end;

  IMyInterface = interface
    procedure setStruct(const tstr: TMyInterface_TestStruct);
  end;
...
\end{verbatim}

%-------------------------------------------------------------------------------

\subsection{FPC mapping}
A special parser can read \fpc source codes and translate all interface
declarations together with their referenced type and constant declarations back
to IDL. This mapping process is simply the reverse process of the IDL to \fpc
mapping.

%===============================================================================
\chapter{Special Notes on \fpc Beta}
\label{ChapterFPCBeta}
Unfortunately, some of the compiler features needed by FlexORB are not available
in the compiler at the time of this writing. They are expected to appear quite
soon after \fpc 1.0 has been released; in the meantime the FlexORB beta
implementation behaves somewhat different than described in this document:
\begin{itemize}
\item There is not support for the \texttt{interface} construct yet. Instead,
      normal classes (\texttt{class} constructs) are used to emulate interfaces.
\item Multiple Inheritance is emulated by FlexORB, too: The object
      implementations have to register theirselves once per run, FlexORB will
      create all the necessary glue code and virtual method tables from this
      informations.
\item It does not yet create the run-time type information (RTTI) necessary for
      the run-time interface inspection capabilities of CORBA. FlexORB will have
      to use other means of getting the RTTI data in the meantime.
%### Update as soon as possible:
      (Wether this
      will be realized by reading external RTTI files or by an extended version
      of the registering procedure necessary for the Multiple Inheritance
      support, has to be defined ASAP.)
\end{itemize}

\end{document}
